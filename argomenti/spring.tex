\chapter{Spring Core e Spring Boot}
\section{Spring}
Spring è un framework basato su due concetti chiave: Dependency Injection e Inversion of Control.

\section{Bean}
Per realizzarlo si serve di un elemento fondamentale, cioè i Bean. Possono essere realizzati tramite XML o annotazioni, in particolare utilizzando:

\begin{itemize}
    \item \textbf{@Configuration}: indica che la classe dichiara uno o più metodi Bean, da usare prima della classe;
    \item \textbf{@ComponentScan}(value=package): specifica i package che vogliamo vengano scannerizzati;
    \item \textbf{@Bean}: il componente chiave e peculiarità di Spring.
\end{itemize}

Una volta dichiarato il nostro metodo Bean, sarà possibile ottenere oggetti recuperandoli dal contesto utilizzando il metodo getBean, utilizzando la classe o il nome del bean; utilizzando ciò è possibile ridurre la strettezza dell'accoppiamento riducendo quindi l'effetto a cascata di dover modificare tante classi per la modifica di una sola classe.

E' possibile specificare alcuni parametri per il Bean tra i quali lo scope (in caso non venga definito si tratterà di un singleTon, cioè ritornerà lo stesso oggetto a tutte le chiamate).

Un'altra importante annotazione utilizzata è \textbf{@Autowired}.

\section{Spring Boot}
E' un'estensione del framework Spring per automatizzare diverse configurazioni che altrimenti andrebbero impostati manualmente, tra cui installazione del server (fornisce infatti un server HTTP già incluso e utilizzabile, alcune librerie spring eccetera \dots).
