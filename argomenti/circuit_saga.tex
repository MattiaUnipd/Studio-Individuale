\chapter{Circuit Breaker e Saga Pattern}
\section{Circuit Breaker}
\subsection{Cos'è?}
Il Circuit Breaker è un Pattern utilizzato per prevenire che il fallimento di un servizio generi un `effetto a cascata' sugli altri servizi che interagiscono con esso.

\subsection{Quando è utile e come funziona}
La necessità di un Circuit Breaker si fa sentire quando un servizio diventa non disponibile o avente latenza così alta da risultare de facto inutilizzabile: questo potrebbe causare uno spreco di risorse - talvolta anche importante - da parte di un servizio che cerca di chiamarlo e rimane in attesa di risposta, che in alcuni casi potrebbe portare anche questo secondo servizio ad essere inutilizzabile.

In casi come questo entra in gioco il Circuit Breaker, che in caso registri un numero eccessivo di fallimenti (banalmente, quando supera un certo treshold arbitrario) scatta e fa partire un periodo di timeout, dove tutte le richieste a tale servizio falliranno immediatamente. Alla fine del periodo di timeout lascerà passare un numero limitato di richieste: in caso queste arrivino con successo permetterà di riprendere normali operazioni, altrimenti fa ripartire il periodo di timeout.

\section{Saga Pattern}
Si tratta di un pattern utilizzato per mantenere dati coerenti durante operazioni che potrebbero richiedere l'impiego di servizi diversi: un esempio banale potrebbe essere il pagamento di un ordine in un e-commerce, dove devo non solo aggiornare la tabella degli ordini con l'oggetto appena acquistato, ma anche quella dei pagamenti, e ho bisogno che i dati rimangano coerenti attraverso l'intera operazione.

Il Saga Pattern fa cioè in modo che l'ordine di esecuzione delle transazioni di una saga venga rispettato e che il loro effetto non venga compromesso da operazioni intermedie.

Ci sono due tipi principali di approccio: 
\begin{itemize}
    \item choreography => nessun coordinatore, i servizi portano a termine la saga collaborando tra loro
    \item orchestration => un orchestratore fa da controllo centrale per la corretta riuscita della saga
\end{itemize}